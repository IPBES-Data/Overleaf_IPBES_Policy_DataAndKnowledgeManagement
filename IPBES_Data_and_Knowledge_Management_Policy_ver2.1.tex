\documentclass{article}

\title{IPBES DATA AND KNOWLEDGE MANAGEMENT POLICY}

\usepackage{hyperref}
\hypersetup{
    colorlinks=true,      % Enables colored links instead of boxes
    urlcolor=blue,        % Blue color for URLs (underlined)
    linkcolor=red,        % Red color for internal links (e.g., \ref, \gls)
    citecolor=green,      % Green color for citations (\cite)
}

\usepackage{geometry}
\geometry{a4paper, margin=1in}
\usepackage{enumitem}
\usepackage{graphicx}
\usepackage{amsmath}
\usepackage{amssymb}
\usepackage{titlesec}
\usepackage{textcomp}
\usepackage{footnote}
\usepackage{lipsum}
\makesavenoteenv{tabular}
\usepackage{enumitem}
\usepackage{fontawesome5} % For the email (envelope) icon

\usepackage[colorinlistoftodos]{todonotes}
\usepackage{pifont} % for ✔

\usepackage[toc]{glossaries}
\renewcommand{\glossarysection}[2][]{}
\renewcommand*{\glstextformat}[1]{\textit{\textcolor{blue}{#1}}}

% \let\oldnewacronym\newacronym
% \renewcommand{\newacronym}[2][]{%
%   \oldnewacronym[type=main,#1]{#2}%
% }

%%%%%%%%%%%%%%%%%%%%%%%%%%%%%%%%%%%%%%%%%%%%%%%%%%%%%%%%%%%%%%%%%%%%%%%%%%
%%%%%%%%%%%%%%%%%%%%%%%%%%%%%%%%%%%%%%%%%%%%%%%%%%%%%%%%%%%%%%%%%%%%%%%%%%
%% The glossaries pckage should be loaded in the main document by using %%
%%                                                                      %%
%% \usepackage[toc,acronym]{glossaries}                                 %%
%% \makeglossaries                                                      %%
%%%%%%%%%%%%%%%%%%%%%%%%%%%%%%%%%%%%%%%%%%%%%%%%%%%%%%%%%%%%%%%%%%%%%%%%%%
%%%%%%%%%%%%%%%%%%%%%%%%%%%%%%%%%%%%%%%%%%%%%%%%%%%%%%%%%%%%%%%%%%%%%%%%%%

%%%%%%%%%%%%%%%%%%%%%%%%%%%%%%%%%%%%%%%%%%%%
%%%%%%%%%%%%%%%%%%%%%%%%%%%%%%%%%%%%%%%%%%%%
%%% acronym entries                       %%
%%%%%%%%%%%%%%%%%%%%%%%%%%%%%%%%%%%%%%%%%%%%
%%%%%%%%%%%%%%%%%%%%%%%%%%%%%%%%%%%%%%%%%%%%

\newacronym[
    description={To be added} %description
    plural=IDEs,
    firstplural=Integrated Development Environments (IDEs)    
    ]
    {ide} % id
    {IDE} % short
    {Integrated Development Environment} %long

\newacronym[
    description={To be added} %description
    ]
    {ai-cop} % id
    {AI CoP} % short
    {Artificial Intelligence Code of Practice} %long

\newacronym[
    description={In the context of this document, \textit{AI} is limited to generative Artificial Intelligence (short: \textit{AI}). It can, for example, generate text, images, figures, videos or computer code based on prompts provided by the user. AI models are trained with large amounts of data. These models and prompts can be hidden behind familiar (non-\textit{AI}) looking interfaces like literature search interfaces or spell and grammar checkers. In this document, AI refers to all tools directly interacting with a generative \textit{AI} (e.g. chatbots) or other tools which use generative \textit{AI} in the background.}
    plural=AIs,
    firstplural=Artificial Intelligences (AIs)    
    ]
    {ai} % id
    {AI} % short
    {Artificial Intelligence} %long

\newacronym[
    description={To be added}
    ]
    {llm}
    {LLM}
    {Large Language Model}

\newacronym[
    description={An independent intergovernmental body established by States to strengthen the science-policy interface for biodiversity and ecosystem services for the conservation and sustainable use of biodiversity, long-term human well-being and sustainable development. It was established in Panama City, on 21 April 2012 by 94 Governments.  It is not a United Nations body.}
    ]
    {ipbes}
    {IPBES}
    {Intergovernmental Science-Policy Platform on Biodiversity and Ecosystem Services}

\newacronym[%
    description={ (short: CARE principles): A set of guiding principles for indigenous data governance focusing on appropriate use and reuse of indigenous \textit{data and knowledge}. See here for specifications. CARE is people and purpose-oriented and includes the principles of Collective Benefit, Authority to Control, Responsibility and Ethics. These guiding principles should also be applied to the management of \textit{knowledge}. }%
    ]
    {care}
    {CARE}
    {CARE Principles for Indigenous Data Governance}%


\newacronym[
    description={(A \textit{data and knowledge management report} is a formal document containing information concerning the handling of \textit{data and knowledge} during and after the finalization of the \gls{researchproject}. It should be drafted at the beginning of the project and be maintained and updated during the whole duration of the assessment to be kept up to date. It describes: The \textit{data and knowledge} that will be created; the process of how the \textit{data and knowledge} have been created, including references to the original data sources, scripts, and software used (see "\gls{workflow}" below); all additional information to make the process of the \textit{data and knowledge} generation as transparent and reproducible as possible; access to the \textit{data and knowledge} and where the \textit{data and knowledge} will be preserved.},
    plural=DMRs,
    firstplural={Data and Knowledge Management Reports (DMRs)}
    ]
    {dmr}
    {DMR}
    {Data and Knowledge Management Report}

\newacronym[
    description={A Digital Object Identifier as defined in the DOI Handbook. A digital identifier of an object, not an identifier of a digital object. \textit{DOIs} are an ISO standard (ISO 26324-2012) and provide an actionable, interoperable, and persistent link.},
    plural=DOIs,
    firstplural={Digital Object Identifiers (DOIs)}
    ]
    {doi}
    {DOI}
    {Digital Object Identifier}

\newacronym[
    description={A set of guiding principles to make \textit{data} Findable, Accessible, Interoperable, and Reusable (FAIR). See here for specifications. These guiding principles should also be applied to the management of \textit{knowledge}.},
    firstplural={FAIR data principles},
    plural={FAIR data principles}
    ]
    {fair}
    {FAIR}
    {FAIR data principle}

\newacronym[
    description={As noted in the IPBES core glossary: "indigenous and local knowledge systems are social and ecological knowledge practices and beliefs pertaining to the relationship of living beings, including people, with one another and with their environments."},
    firstplural={Indigenous and Local Knowledge systems (ILK systems)},
    plural={ILK systems}
    ]
    {ilk}
    {ILK}
    {Indigenous and Local Knowledge system}

\newacronym[
    description={As noted in the IPBES core glossary: "\textit{indigenous peoples and local communities} (IPLCs) are, typically, ethnic groups who are descended from and identify with the original inhabitants of a given region, in contrast to groups that have settled, occupied or colonized the area more recently."},
    plural=IPLCs,
    firstplural={Indigenous Peoples and Local Communities (IPLCs)}
    ]
    {iplc}
    {IPLC}
    {Indigenous People or Local Community}

\newacronym[
    description={A subsidiary body established by the \textit{IPBES Plenary} which oversees the scientific and technical functions of the Platform; a key role is to select \textit{experts} to carry out assessments.},
    plural=MEPs,
    firstplural={Multidisciplinary Expert Panels (MEPs)}
    ]
    {mep}
    {MEP}
    {Multidisciplinary Expert Panel}

\newacronym[
    description={A working group of domain \textit{experts}, established by the \gls{plenary}, to carry out tasks as defined in the terms of reference under the Platform’s rolling work programme.},
    plural={Task Forces},
    firstplural={Task Forces}
    ]
    {tf}
    {TF}
    {Task Force}

\newacronym[
    description={The \textit{technical support unit} works under the oversight of the \textit{secretariat} to coordinate and administer the activities of \textit{expert} groups in support of the development of deliverables. Technical support units are dedicated to a specific assessment or a \textit{task force}.},
    plural=TSUs,
    firstplural={Technical Support Units (TSUs)}
    ]
    {tsu}
    {TSU}
    {Technical Support Unit}


%%%%%%%%%%%%%%%%%%%%%%%%%%%%%%%%%%%%%%%%%%%%
%%%%%%%%%%%%%%%%%%%%%%%%%%%%%%%%%%%%%%%%%%%%
%%% glossary entries                      %%
%%%%%%%%%%%%%%%%%%%%%%%%%%%%%%%%%%%%%%%%%%%%
%%%%%%%%%%%%%%%%%%%%%%%%%%%%%%%%%%%%%%%%%%%%

\newglossaryentry{promptengineering}{%
    name={prompt engineering},
    description={TO BR ADDED}
}

\newglossaryentry{accessbenefitsharing}{%
    name={access and benefit-sharing},%
    description={ is rooted in the CARE principles and refers to the understanding that data, knowledge or resources should be used in ways that benefit the community or group from which they originate. This principle ensures that the interests of the community are prioritised and that any use of data or resources contributes positively to their well-being and development. }%
}

    
\newglossaryentry{bureau}{%
    name={ Bureau },%
    description={ A subsidiary body established by the \textit{Plenary }which carries out administrative functions. It is made up of representatives nominated from each of the United Nations regions and is chaired by the Chair of IPBES. }%
}
\newglossaryentry{citationsandreferences}{%
    name={ Citations and references },%
    description={ A \textit{citation} within an \gls{ipbesproduct} refers to the source of information to the \textit{data} and metadata supporting IPBES deliverables, and addresses where the information came from. A \textit{reference }includes adequate details about the source of the information making it findable and traceable. }%
}

\newglossaryentry{collectivebenefit}{%
    name={ Collective benefit },%
    description={Benefits for all. These are, in the context of this policy, direct benefits resulting from use of and access to \glspl{ipbesproduct}, and \gls{dataandknowledge} gathered or documented during their production. To this end, \glspl{ipbesproduct} and \gls{dataandknowledge} should, to the maximum extent possible, be openly available and designed so they are accessible to all, to allow scientists, IPBES members, IPLCs and others to use them and consequently derive benefit from them. }%
}

\newglossaryentry{dataandknowledge}{%
    name={ Data and knowledge },%
    description={ In many cases, \textit{data} can not be interpreted without \textit{knowledge} and must be seen as an item together with \textit{knowledge}. \gls{data} and \textit{knowledge} form a continuous spectrum and must not be separated from each other. In a general sense, they consist of individual units of information, which are obtained from observations, measurements, experiences, value systems, etc. They form the basis of monitoring, \textit{research}, assessments, and analysis.  
    }
}

\newglossaryentry{data}{%
    name = {data},
    description = {Data can be of any nature, including among others, spatial or non-spatial, qualitative or quantitative, descriptive, and from all scientific disciplines. This includes information from \textit{indigenous peoples and local communities (IPLC)}.
    }
}

\newglossaryentry{knowledge}{%
    name = {knowledge},
    description = {Knowledge is the understanding gained through experience, reasoning, interpretation, perception, intuition, and learning that is developed as a result of information use and processing. Knowledge is often essential in the interpretation and understanding of associated \gls{data}. }%
}

\newglossaryentry{datadepositpackage}{%
    name={ Data deposit package },%
    description={ the content deposited in a long-term repository. Each \textit{data deposit package} has a\textit{ DOI}. A \textit{data deposit package} consists of at least the \textit{data and knowledge} itself and the \textit{data and knowledge management repor}t describing the \textit{data} as outlined above, unless the data is a final product such as an assessment. }%
}

\newglossaryentry{externaldata}{%
    name={ External data and knowledge },%
    description={ (hereafter "external data) \gls{externaldata} is \textit{data }and \gls{knowledge} which has been generated outside of IPBES and \textit{IPBES products,} and is available and published in peer-reviewed journals, grey literature or other sources or available as \textit{indigenous and local knowledge} (ILK). These products of external entities are typically the input for \textit{research} within IPBES. IPBES is not responsible for any preservation of these products; however if the Platform documents any existing knowledge, IPBES is responsible for preserving the documentation upon agreement with the knowledge holders such as IPLCs in the case of ILK. }%
}



\newglossaryentry{ipbesexpert}{%
    name={ IPBES expert },%
    plural = {IPBES experts},
    description={Any person conducting \textit{research} in the context of IPBES, in particular, its assessments and \glspl{tf}. \textit{IPBES Experts} also include \gls{mep} members advising in the context of the preparation of \glspl{ipbesproduct}. }%
}

\newglossaryentry{ipbesproduct}{%
    name={IPBES product},%
    plural={IPBES products},%
    description={ Factual records produced by and within IPBES which can be used as primary sources for scientific research and which are required to validate its results. They vary according to the area of \textit{knowledge }and may be contained in textual, non-textual, digital or physical formats including documents, spreadsheets, databases, maps, statistics, diaries, questionnaires, transcriptions, audio files, video, photographs, images, models, algorithms, scripts, log files, simulation software, methodologies and \textit{workflows}, operating procedures, standards, protocols and any new products developed in the future in digital or physical formats. \textit{Knowledge }and the data generated by applying this \textit{knowledge} to \textit{external data} are referred to as\textit{ IPBES products}. }%
}

\newglossaryentry{milestone}{%
    name={ Milestone },%
    description={ A significant step towards the completion of the overall goal of a \gls{researchproject} which warrants its long-term storage. In the case of assessments, this would include the completion of zero-, first- and second-order drafts of the chapters of the assessment as well as their final versions. Other defined \textit{milestones }can be added if deemed necessary. For other \gls{researchproject}, \textit{milestones }should be defined in the planning phase of the \gls{researchproject}. }%
}

\newglossaryentry{plenary}{%
    name={ Plenary },%
    description={ The decision-making body of IPBES comprising representatives of members of IPBES. }%
}

\newglossaryentry{research}{%
    name={ Research },%
    description={ \gls{research} refers to all activities within IPBES which collect, measure, aggregate, process, integrate, or analyse \textit{data}, including \textit{indigenous and local knowledge} (\textit{ILK}), or newly generated \textit{data}\textit{and knowledge}. It also includes the documentation of \textit{ILK} during dialogue workshops or other activities with IPLCs. The term "\textit{research}" used in this policy refers to the process of preparation of \textit{IPBES products.} }%
}

\newglossaryentry{researchproject}{%
    name={ Research project },%
    description={ A chapter in an assessment, which is coordinated by coordinating lead author(s) and conducted by lead authors and/or fellows; a task associated with a single or multiple \textit{IPBES product(s)}. }%
}

\newglossaryentry{secretariat}{%
    name={ Secretariat },%
    description={ The \textit{secretariat }of the Platform. }%
}

\newglossaryentry{stakeholder}{%
    name = { Stakeholders in IPBES },%
    plural = {stakeholders},
    description={Stakeholders in IPBES (short: stakeholders): In the context of the work programme and the IPBES stakeholder engagement policy as set out in decision IPBES-3/4, \textit{stakeholders} act as both contributors and end-users of the Platform and are individual scientists or knowledge holders, and also institutions, organizations or groups working in the field of biodiversity (See IPBES/3/16), which: Contribute to the activities of the work programme through their experience, expertise, \textit{knowledge, data}, information and capacity-building experience; Use or benefit from the outcomes of the work programme; Encourage and support the participation of scientists and knowledge holders in the work of the Platform. }%
}

\newglossaryentry{unescoopenscience}{%
    name={ UNESCO Open Science recommendations },%
    description={ An international framework for open science policy and practice as unanimously adopted by UNESCO Member States at the Science Commission plenary at its 41st General Conference. See here for specifications. The recommendation outlines common definitions, shared values, principles and standards for open science at the international level and proposes a set of actions conducive to a fair and equitable operationalization of open science for all at the individual, institutional, national, regional and international levels. }%
}

\newglossaryentry{workflow}{%
    name={workflow},%
    plural={workflows},
    description={ A repeatable set of steps involved in achieving a goal. Individual steps could be ‘\textit{data and knowledge} gathering’, ‘\textit{data} filtering’, ‘\textit{data} preparation’, ‘\textit{data and knowledge} analysis’, and ‘\textit{data and knowledge} visualisation’. An analytical workflow, for example, represents the transformations made to \textit{data} along the scientific process, including \textit{data} sources, scripts and software used. }%
}


\makeglossaries

% adjust space between number and Header text
\usepackage{tocloft}
\renewcommand{\cftsecnumwidth}{2.5em}  % Adjust this value as needed
\renewcommand{\cftsubsecnumwidth}{2.5em}  % Adjust this value as needed

% Roman Numbers for section numbering
\renewcommand{\thesection}{\Roman{section}}

% Alphabetic (a), (b), ... for subsections
% \usepackage{alphalph}
% \renewcommand{\thesubsection}{(\alphalph{\value{subsection}})}



\usepackage{draftwatermark}
\SetWatermarkText{CONFIDENTIAL DRAFT}
\SetWatermarkScale{1.5}     % Size
\SetWatermarkColor[gray]{0.85}  % Color (light gray)
\SetWatermarkAngle{45}      % Angle

% \author{
%     \LARGE \textbf{Editors} \\[0.5cm]  % This adds "Editors" in bold and slightly spaced from the names
%     %%
%     Rainer M. Krug \href{https://orcid.org/0000-0002-7490-0066}{\includegraphics[width=10pt]{orcid.png}}\ \href{mailto:Rainer@Krugs.de}{\faEnvelope}\ \thanks{Institution Name} \\
%     %%
%     Peter Bates \\ 
%     %%
%     Joy Kumagai \\
%     %%
%     Aidin Niamir
% }
% \renewcommand{\author}{Editors}

% \date{}



\begin{document}

% \maketitle

%%%%%%%%%%%%%%%%%%%%%%%%%%%%%%%%%%%%%%%%%%%%
%%%%%%%% TODO Items
%%%%%%%%%%%%%%%%%%%%%%%%%%%%%%%%%%%%%%%%%%%%

% %%%%%%%%%%%%%%%%%%%%%%%%%%%%%%%%%%%%%%%%%%%%
%%%%%%%% begin Titlepage
%%%%%%%%%%%%%%%%%%%%%%%%%%%%%%%%%%%%%%%%%%%%

\begin{center}

    {\LARGE \textbf{Contributing Authors (in alphabetical order)}} \\[0.5em]
    Douglas Beard, 
    Maral Dadvar \href{https://orcid.org/0000-0002-1351-2561}{\includegraphics[width=8pt]{orcid.png}}\ , 
    Brenda Daly \href{https://orcid.org/0000-0002-3732-8339}{\includegraphics[width=8pt]{orcid.png}}\ , 
    Adriana De Palma \href{https://orcid.org/0000-0002-5345-4917}{\includegraphics[width=8pt]{orcid.png}}\ , 
    Debora Pignatari Drucker \href{https://orcid.org/0000-0003-4177-1322}{\includegraphics[width=8pt]{orcid.png}}\ , 
    Golda Edwin \href{https://orcid.org/0000-0002-7487-3272}{\includegraphics[width=8pt]{orcid.png}}\ , 
    Renske Gudde \href{https://orcid.org/0000-0003-4727-1011}{\includegraphics[width=8pt]{orcid.png}}\ , 
    Kwame Oppong Hackman \href{https://orcid.org/0000-0002-2201-9314}{\includegraphics[width=8pt]{orcid.png}}\ , 
    Jessica Hetzer \href{https://orcid.org/0000-0002-2384-0024}{\includegraphics[width=8pt]{orcid.png}}\ , 
    Rainer M. Krug \href{https://orcid.org/0000-0002-7490-0066}{\includegraphics[width=8pt]{orcid.png}}\ ,
    Milan Mataruga \href{https://orcid.org/0000-0002-9318-7550}{\includegraphics[width=8pt]{orcid.png}}\ , 
    Aidin Niamir \href{https://orcid.org/0000-0003-4511-3407}{\includegraphics[width=8pt]{orcid.png}}\ , 
    Benedict A. Omare \href{https://orcid.org/0000-0003-3704-0332}{\includegraphics[width=8pt]{orcid.png}}\ , 
    Ricardo Pinto-Coelho \href{https://orcid.org/0000-0003-4486-1243}{\includegraphics[width=8pt]{orcid.png}}\ , 
    Hanno Seebens \href{https://orcid.org/0000-0001-8993-6419}{\includegraphics[width=8pt]{orcid.png}}\ , 
    Yanina Sica \href{https://orcid.org/0000-0002-1720-0127}{\includegraphics[width=8pt]{orcid.png}}\ , 
    Aysegül Sirakaya \href{https://orcid.org/0000-0003-3330-1750}{\includegraphics[width=8pt]{orcid.png}}\ , 
    Zheping Xu 
    and
    Mayra Alejandra Zurbaran Nucci \href{https://orcid.org/0000-0003-1231-3859}{\includegraphics[width=8pt]{orcid.png}}\ .
    
    {\LARGE \textbf{Contributing Authors (Previous revisions in alphabetical order)}} \\[0.5em]
    Wouter Addink, Gregoire Dubois, Renske Gudde, Joy Kumagai, Rainer M. Krug, Cornelia Krug, Howard Nelson, Aidin Niamir, Benedict A. Omare, Fatima Parker-Allie, Debora Pignatari Drucker, and David Thau.
    
    \vspace{1.5em}
    
    {\LARGE \textbf{DOI}} \\[0.5em]
    \href{https://doi.org/10.5281/zenodo.3551078}{10.5281/zenodo.3551078}
    
    \vspace{1.5em}
    
    {\LARGE \textbf{Version}} \\[0.5em]
    Version 2.1.DRAFT
    
    \vspace{1.5em}
    
    {\LARGE \textbf{Date}} \\[0.5em]
    \today
    
    \vspace{1.5em}
    
    {\LARGE \textbf{Suggested Citation}} \\[0.5em]
    IPBES (2024): IPBES Data and Knowledge Management Policy ver. 2.1, Krug, R.M., Omare, B., and Niamir, A. (eds.) IPBES secretariat, Bonn, Germany. DOI: \href{https://doi.org/10.5281/zenodo.3551078}{10.5281/zenodo.3551078}
    
    \vspace{1.5em}
    
    {\LARGE \textbf{Latest Version}} \\[0.5em]
    To get the latest version of the policy visit \href{https://doi.org/10.5281/zenodo.3551078}{https://doi.org/10.5281/zenodo.3551078}
    
    \vspace{1.5em}
    
    {\LARGE \textbf{Inquiries}} \\[0.5em]
    Inquiries should be directed to \href{mailto:the.email@somewhere}{the.email@somewhere}

\end{center}

\newpage
%%%%%%%%%%%%%%%%%%%%%%%%%%%%%%%%%%%%%%%%%%%%
%%%%%%%% end Titlepage
%%%%%%%%%%%%%%%%%%%%%%%%%%%%%%%%%%%%%%%%%%%%

% \listoftodos

\tableofcontents

\newpage

%%%%%%%%%%%%%%%%%%%%%%%%%%%%%%%%%%%%%%%%%%%%
%%%%%%%% TODO Items
%%%%%%%%%%%%%%%%%%%%%%%%%%%%%%%%%%%%%%%%%%%%

% input(todoss)

%%%%%%%%%%%%%%%%%%%%%%%%%%%%%%%%%%%%%%%%%%%%
%%%%%%%% The actual policy
%%%%%%%%%%%%%%%%%%%%%%%%%%%%%%%%%%%%%%%%%%%%


\section{Preamble}

The \gls{plenary} of the \gls{ipbes}, in section II of its decision IPBES-2/5, established a \gls{tf} on knowledge and data for the period of its first work programme 2014‒2018. In decision IPBES-3/1, section II, the \gls{plenary} approved the data and information management plan set out in annex II to the same decision. At its seventh session, in decision IPBES-7/1, section II, the \gls{plenary} adopted the rolling work programme of the Platform up to 2030, which included among its six objectives objective 3 (a), advanced work on knowledge and data. In section IV of its decision IPBES-7/1, the \gls{plenary} recalled the establishment of the \gls{tf} and extended its mandate for the implementation of objective 3 (a) of the rolling work programme of the Platform up to 2030, in accordance with the revised terms of reference set out in annex II to the same decision, and requested the \gls{bureau} and \gls{mep}, through the \gls{secretariat} to constitute the \gls{tf} in accordance with the terms of reference. According to its terms of reference, the \gls{tf} on knowledge and data oversees and takes part in the implementation of objective 3 (a) of the rolling work programme up to 2030 and acts in accordance with relevant decisions by the \gls{plenary} and its subsidiary bodies. Its mandate includes, among other things, to guide the \gls{secretariat}, including the dedicated \gls{tsu} in the management of the \gls{data}, information and \gls{knowledge} used in \glspl{ipbesproduct}, including the development of the web-based infrastructure, to ensure their long-term availability, reusability and data interoperability. The purpose of the policy is to provide overarching guidance on the management of data and knowledge to current and upcoming assessments and the work of task forces regarding \glspl{ipbesproduct}. The policy is grounded in the principles of open science, \textit{accessibility}, and building knowledge through partnerships.accessibility

\section{Introduction}

This policy builds on the data and information management plan approved by the IPBES \gls{plenary} set out in annex II to decision IPBES-3/1, and the earlier revision of the IPBES data and knowledge management policy welcomed by the IPBES Plenary at its 9th session, set out in IPBES/9/INF/14, appendix II to the annex. It builds in particular on the activity "reviewing and developing data and metadata guidelines", and is grounded in its principles for "managing knowledge, information, and data" in the Platform, in particular accessibility, and open science.


\begin{itemize}
    \item \textbf{Open science}. The open science approach promotes the generation of \gls{knowledge} through collaboration based on free and open access to \gls{knowledge}, information, and \gls{data}. Open science, therefore, ensures that the work of all the \glspl{ipbesexpert} and \glspl{stakeholder} involved is fully recognised, properly attributed, documented, and preserved. Adoption of the open science approach and principles means a significant cultural change in the ways in which science is done, and scientific results and underlying \gls{data} are shared publicly by authors, journals and research organisations and thus made relevant to society. In the context of the Platform, the open science approach brings very significant advances in \gls{data} integration, analysis and interpretation and could ultimately lead to a better understanding of nature and its contribution to a good quality of life. Two key aspects of open science are "\textit{accessibility}" and "inclusivity and \gls{collectivebenefit}".

    \item \textbf{Accessibility.} A core value of the Platform is free and open access to its deliverables and to the material on which they are based. Consequently, the policy will aim for open, sustained access to \gls{data} and information sources for its deliverables (e.g., in the scientific literature) with as few restrictions as possible; enforce the use of common and \gls{accessible}, non-propriaraty file formats in the Platform's deliverables; emphasise the need to make data and knowledge available;  and facilitate multilingual discovery and sharing of \gls{data} and \gls{knowledge} wherever possible. The Platform acknowledges that making \gls{data} and information \gls{accessible} may not always mean it is always \gls{accessible} to all IPBES members and\glspl{stakeholder}, including \glspl{iplc}, as technical, economical, political or any other reasons may limit accessibility, but not its findability.
    
    \item \textbf{Inclusivity and \gls{collectivebenefit}.} Cooperation in \gls{knowledgesynthesis} and broad acceptance of the resulting \glspl{ipbesproduct} is essential for the Platform to fulfil its mandate. For the acceptance and cooperation of \glspl{stakeholder} and \gls{dataandknowledge} holders, inclusivity in all stages of the \gls{knowledgesynthesis} is essential, and IPBES takes many steps to try to enhance participation beyond only scientific researchers. IPBES acknowledges the richness of diverse knowledge systems and epistemologies and diversity of knowledge holders and producers, including \gls{ilk} and \glspl{iplc}; also outlined in the \gls{unescoopenscience}\footnote{\href{https://en.unesco.org/science-sustainable-future/open-science/recommendation}{https://en.unesco.org/science-sustainable-future/open-science/recommendation}}.

    \item \textbf{Responsible and ethical use of advances in data technologies}. The ethical and responsible use of advances in data technologies, in particular \gls{ai}, ensures the appropriate use of these technologies and fosters trust in the Platform. The use of \gls{ai} should be grounded in transparency, accountability, fairness and respect for human rights and diverse knowledge systems. The values of inclusivity, accessibility and \gls{collectivebenefit} should guide the use of \gls{ai}. This approach supports the equitable advancement of knowledge and ensures that the use of \gls{ai} benefits \glspl{ipbesproduct} following this policy.
\end{itemize}

Thus, this policy follows regarding \gls{dataandknowledge} management within IPBES the UNEP’s over-arching strategies, policies and guidelines \footnote{\href{https://www.unep.org/about-un-environment/policies-and-strategies}{https://www.unep.org/about-un-environment/policies-and-strategies}}, to the maximum extent possible within the mandate of the Platform, allowing IPBES members and \glspl{stakeholder}, \glspl{iplc} and other interested parties to use and access \glspl{ipbesproduct}, and \gls{data} gathered or documented during their production and consequently derive benefit from them. IPBES has recognised the importance of \gls{ilk} to the conservation and sustainable use of ecosystems, and the procedures and protocols to be used with regard to \gls{ilk} and \glspl{iplc} have been developed in detail. From its inception, the Platform aims to enhance inter-relationships and complementarities between different knowledge systems.

\section{Objectives}

To fulfil its function to generate transparent assessments, IPBES is committed to implementing \gls{dataandknowledge} management procedures that are discipline-appropriate, practical, cost-effective and sustainable, and supportive of its objectives. This data and knowledge management policy is the primary reference document for IPBES \gls{dataandknowledge} management. It serves to ensure that \gls{dataandknowledge} is managed correctly and consistently, and is maintained to the highest possible standard. The data and knowledge management policy has the following objectives:


\begin{enumerate}[label=(\alph*)]
    \item To ensure that \gls{dataandknowledge} produced during IPBES \gls{knowledgesynthesis} activities follow legal obligations and the \gls{fair} and \gls{care} principles to the fullest extent possible within the mandate of the Platform;
    
    \item To ensure that \glspl{ipbesproduct}, to the maximum extent possible within the mandate of the Platform, are openly available and designed so they are accessible; allowing all scientists, IPBES members, \glspl{iplc} and other \glspl{stakeholder} to use them and consequently derive and share the benefits from them;

    \item To give full consideration to existing and evolving principles and guidelines for the applicability and ethical and responsible use of advances in data technology including \gls{ai};

    \item To provide guidance to all \glspl{entity}, including \gls{secretariat} and \glspl{ipbesexpert}, to fulfil their responsibilities with respect to management, handling, preservation, and distribution of \gls{dataandknowledge}, generated \gls{data}, and \glspl{ipbesproduct} within the Platform;
    
    \item To provide a suggested \gls{workflow} for long-term storage and preservation of \glspl{ipbesproduct} to the \glspl{ipbesexpert};
    
    \item To promote the usage of open-source software to enable users to recreate and use \glspl{ipbesproduct} without limitations.

    \item To guide the \glspl{ipbesexpert} to fulfil their responsibilities to develop the necessary \glspl{dmr} which fulfil the requirements of this policy;
\end{enumerate}


\section{General Principles}

To the fullest extent possible within the mandate of the Platform, \glspl{ipbesproduct} and the associated \gls{knowledgesynthesis} should be managed following the \gls{fair} and \gls{care} principles throughout their life cycle in line with the commitment to open science and accessibility.

\glspl{ipbesproduct} and associated \gls{knowledgesynthesis} relating to \glspl{iplc} or incorporating \gls{ilk}, will follow the \gls{ipbes} approach to recognising and working with \gls{ilk} (Annex II to decision IPBES-5/1).

\glspl{ipbesproduct} which follow the \gls{fair} and \gls{care} principles to the fullest extent possible within the mandate of the Platform are essential for fulfilling the functions of \gls{ipbes}; to perform regular and timely assessments of \gls{knowledge} on nature, its contribution to good quality of life, and their interlinkages, in a transparent and \textit{reproducible} manner.

In the management, handling, and delivery of \glspl{ipbesproduct}, international, regional, and national law should be complied with, which includes \gls{accessbenefitsharing}, sovereign rights over genetic resources, rights of privacy, intellectual property rights, \gls{data} governance regulations, and duties of confidentiality as well as other legal obligations to which IPBES has agreed as binding upon IPBES and that fall outside the scope of this policy. \glspl{ipbesproduct} should be anonymised, if necessary, before long-term storage and publication.

IPBES experts are encouraged to take advantage of advances in \gls{data} technologies and to make best use of these advances in the preparation of \glspl{ipbesproduct}. These technologies should be used responsibly, with particular attention to ethics, transparency and underlying biases. The use of \gls{ai} within IPBES should be precautionary, following the Principles for the Ethical Use of Artificial Intelligence in the United Nations System\footnote{\href{https://unsceb.org/principles-ethical-use-artificial-intelligence-united-nations-system }{https://unsceb.org/principles-ethical-use-artificial-intelligence-united-nations-system }} and the UNESCO Recommendation on the Ethics of Artifical Intelligence\footnote{\href{https://www.unesco.org/en/articles/recommendation-ethics-artificial-intelligence }{https://www.unesco.org/en/articles/recommendation-ethics-artificial-intelligence }}.

IPBES is committed to providing guidance to all IPBES \glspl{ipbesexpert} to ensure that they are aware of and follow IPBES procedures, which aim to follow the \gls{fair} and \gls{care} principles and responsible use of advances in \gls{data} technologies to the fullest extent possible within the mandate of the Platform.

\section{Application}

IPBES will apply this \gls{data} and \gls{knowledge} management policy to all new and ongoing \glspl{ipbesproduct} and related \gls{knowledgesynthesis}. The policy should be reviewed at least every 2 years by the \gls{tf} on \gls{dataandknowledge} management to adapt to new developments in these principles, taking full account of existing and evolving principles and guidelines on the applicability and ethical use of advances in \gls{data} technology. Exceptions and deviations to this policy have to be agreed upon with the \gls{secretariat} and the \gls{tsu} on \gls{dataandknowledge} management and documented in the corresponding \gls{dmr}.

\section{Scope}

This policy applies to all \glspl{ipbesproduct}. \glspl{ipbesexpert} are required to abide by the terms and conditions agreed with third parties. IPBES also recognises that such third parties’ policies are evolving and that these may require higher levels of \gls{data} accessibility and dissemination in the future.

\section{Compliance and enforcement}

Compliance with the data and knowledge management policy is mandatory for all \glspl{entity} involved in the preparation of the \glspl{ipbesproduct}. Compliance will be monitored by the \gls{tsu} on \gls{dataandknowledge} management. Products will not be accepted as \glspl{ipbesproduct} unless they comply with this policy. 
The roles and responsibilities for the implementation of this policy and the production of \glspl{ipbesproduct} compliant and within all \glspl{task} are distributed among all \glspl{entity} according to their roles. The detailed roles and responsibilities are outlined in Appendix~\ref{sec:app_roles}.

\section{Provisions on Data and Knowledge Management Reporting}


\begin{enumerate}[label=(\alph*)]
    \item A \gls{dmr} is expected for each \gls{task}. This can be achieved by a single \gls{dmr} covering all the analyses and work for an entire \gls{task} or by seperate \glspl{dmr} for each sub-\gls{task}, such as an analysis or the development of one figure, within a \glspl{task};

    \item The \gls{dmr} can consist of the report and if applicable multiple additional documents. When it contains multiple documents, the actual \gls{dmr}, the main document, can be considered an overview document which provides transparency of the methods, with the additional documents providing more technical background for \textit{reproducibility}.

    \item The \gls{dataandknowledge} management should comply with this policy. If this is not possible, exceptions and justifications need to be specified in the \gls{dmr} and be approved by the \gls{tsu} on data and knowledge;

    \item It is the responsibility of the \glspl{ipbesexpert} to ensure that the \gls{dmr} is created, maintained, and updated throughout the life cycle of the \gls{task} and submitted to the associated \gls{tsu} in time for its first \gls{milestone};

    \item The  \gls{tsu} on \gls{dataandknowledge} provides support and, where appropriate guidelines and examples (\gls{dmr}  template), to the \glspl{ipbesexpert} to make sure that \gls{fair} and \gls{care} principles are followed, within the mandate of the Platform, for \gls{dataandknowledge} management and documentation in the \glspl{dmr}. This includes the use of open file formats suitable for long-term storage and retrieval, reusability and interoperability of the \gls{data};

    \item The \gls{tf} and \gls{tsu} on \gls{ilk} provide support and, where appropriate guidelines and examples, to the \glspl{ipbesexpert} to make sure that the \gls{fair} and \gls{care} principles are followed, to the fullest extent possible within the mandate of the Platform, for \gls{dataandknowledge} management and documentation in the \glspl{dmr}, where such \gls{data} relates to \gls{ilk} or \glspl{iplc};

    \item The IPBES \gls{secretariat} provides information about recommended long-term, and to the extent possible certified, open \gls{data} repositories which provide \glspl{doi}.
\end{enumerate}

\section{Provisions on accessibility, inclusivity, and Collective Benefit}

\begin{enumerate}[label=(\alph*)]
    \item \glspl{ipbesproduct} should be preserved including a \gls{doi} for each \gls{milestone} of a \gls{task};

    \item \glspl{ipbesproduct} in or associated with an assessment or other \glspl{ipbesproduct}, including \glspl{dmr}, should be made openly \gls{accessible} in a form that follows this policy at the latest one calendar month after the approval/acceptance of the assessment, or approval or acknowledgement of other \glspl{ipbesproduct} by the \gls{plenary}. \gls{dataandknowledge} related to \glspl{milestone} should also be made \gls{accessible}, as far as confidentiality rules allow for this. Embargo periods are possible but need to be approved by the \gls{tf} on \gls{dataandknowledge} management. Restricted access to the IPBES \gls{data} underlying \glspl{ipbesproduct} is only allowed under special circumstances and needs to be approved by the  \gls{tsu} on \gls{data} and \gls{knowledge} management;

    \item Applicable legal, ethical, privacy and confidentiality, and \gls{data} governance requirements need to be followed and \gls{data} generated, if deemed necessary, anonymised before preservation;

    \item Tools for the finding, processing and final presentation of any \gls{data} should be used in a responsible and ethical manner which is following this policy. No tools should be used without critical assessment of the methods and results;

    \item The management, handling, and delivery of the materials from \glspl{iplc} adhere to the \gls{fair} and \gls{care} principles to the fullest extent possible as outlined in this policy, as well as to other binding conditions outside this policy in accordance with international, regional, and national law;

    \item \glspl{ipbesproduct} and their metadata are released with a clear and \gls{accessible} \gls{data} use license: allowed licenses for the \glspl{ipbesproduct} are Creative Commons By Attribution (CC-BY\footnote{\href{https://creativecommons.org/licenses/by/4.0/}{https://creativecommons.org/licenses/by/4.0/} or newer versions} or licenses equivalent to these, which permits users to copy, distribute and transmit work, adapt work under the condition that the user must attribute the work in the manner specified by IPBES. Divergent licences need to be approved by the  \gls{tsu} on \gls{data} and \gls{knowledge} management;

    \item All \gls{knowledgesynthesis} within IPBES have to be conducted in accordance with agreements with the holders of the \gls{dataandknowledge} regarding use, reuse, presentation, and procedures. Communication with \glspl{iplc} needs to be maintained over the whole \gls{task} cycle to the extent possible within the mandate of the Platform and \gls{ipbes} mandate to provide feedback to the \gls{knowledge} holders as well as include feedback from the \gls{knowledge} holders in the \gls{knowledgesynthesis}, as outlined in the IPBES approach to recognising and working with \gls{ilk} as set out in annex II to decision IPBES-5/1\footnote{\href{https://ipbes.net/sites/default/files/inline/files/ipbes_ilkapproach_ipbes-5-15.pdf}{https://ipbes.net/sites/default/files/inline/files/ipbes\_ilkapproach\_ipbes-5-15.pdf}};

    \item \glspl{ipbesproduct} will be made available and \gls{accessible} to the holders of the \gls{dataandknowledge}, in line with agreed terms documented during ILK dialogue workshops or other activities, with due consideration to the \gls{fair} and \gls{care} principles.
\end{enumerate}

\section{Provisions on the use of advances in data technologies}
\begin{enumerate}[label=(\alph*)]
    \item \gls{ipbes} does not limit the use of advances in data technology, as long as they are used in a way which is compliant with this policy
    \item Although most \gls{ai} systems present little to no risk and offer solutions to various societal challenges, some systems introduce risks that need to be addressed to prevent negative consequences.
    \item When using \gls{ai} in \glspl{task}, the \gls{ethicaluse} and \gls{responsibleuse} must be guaranteed
    \item Further details as well as scenarios of \gls{ai} use in an ethical as well as responsible way are detailed in the IPBES AI Code of Practice in Appendix \ref{sec:aicop}.
\end{enumerate}




\appendix

\section{Glossary}
\label{sec:glossary}

\printglossary

\section{Roles and responsibilities}
\label{sec:app_roles}

\begin{enumerate}[label=(\alph*)]
    \item \textbf{\gls{bureau} and \gls{mep}}
    \begin{itemize}
        \item Will review any changes to the policy as proposed by the \gls{tf} on knowledge and data and consider these for approval.
    \end{itemize}

    \item \textbf{\gls{secretariat}}
    \begin{itemize}
        \item Will execute, under the guidance of the \gls{tf} on data and knowledge management and in cooperation with the  \gls{tsu} on data and knowledge management, the development and maintenance of the guidelines, tutorials, \glspl{workflow} and examples to enable \glspl{ipbesexpert} to implement these policies;
        \item Will maintain an accurate, up-to-date and \gls{accessible} list of \gls{citationsandreferences} (including rich metadata), and links to \gls{externaldata}, \gls{knowledge} and generated \gls{data} as used for and in the \glspl{ipbesproduct}.
    \end{itemize}

    \item \textbf{\gls{tf} on data and knowledge management}
    \begin{itemize}
        \item Will develop and provide guidelines and examples for \gls{dataandknowledge} management and guide the development and maintenance of these;
        \item Will advise the \gls{tsu} on data and knowledge management;
        \item Will review the policy at least every 2 years;
        \item Will determine procedures that comply with legal requirements, fulfil \gls{fair} and \gls{care} principles, including through collaborating with the ILK task force, to the fullest extent possible within the mandate of the Platform;
        \item Will review the annexes, guidelines, tutorials, \glspl{workflow}, and examples related to this policy on a yearly basis to identify gaps and implement new developments.
    \end{itemize}

    \item \textbf{\gls{tf} on \gls{ilk}}
    \begin{itemize}
        \item Will work with the \gls{tf} on knowledge and data to review the policy with regard to aspects relating to IPLCs and ILK at least every 2 years, including to determine procedures that fulfil \gls{fair} and \gls{care} principles to the fullest extent possible within the mandate of the Platform;
        \item Will review the guidelines, tutorials, \glspl{workflow}, and examples related to this policy, particularly with regard to aspects relating to IPLCs and ILK, on a yearly basis to identify gaps and implement new developments.
    \end{itemize}

    \item \textbf{\gls{tsu} on data and knowledge management}
    \begin{itemize}
        \item Will provide support, advice, and participate in efforts from the \gls{tf} on data and knowledge management, and execute, under guidance of the \gls{tf} on data and knowledge management, and in cooperation with the \gls{secretariat}, the development and maintenance of the guidelines and tutorials, \glspl{workflow}, and examples to enable IPBES to implement this policy;
        \item Will review the\glspl{dmr} from the corresponding assessment \glspl{tsu} so that they follow the data and knowledge management policy;
        \item Will make sure that the assessment \glspl{tsu} fulfil their responsibilities as outlined in this policy and the \glspl{dmr} and will collect the metadata of the \glspl{ipbesproduct} from the \glspl{tsu}, so that they can be \gls{accessible} and searchable;
        \item Will provide guidance to \glspl{ipbesexpert} on the responsible use of advances in \gls{data} technologies.
    \end{itemize}

    \item \textbf{\gls{tsu} on \gls{ilk}}
    \begin{itemize}
        \item Will provide support, advice, and participate in efforts from the \gls{tf} on data and knowledge management to develop guidelines and examples for \gls{dataandknowledge} management and reporting working with \gls{ilk} and \gls{care} principles;
        \item Will provide support, advice, and participate in efforts to implement these guidelines in IPBES assessments and other processes;
        \item Will review the \glspl{dmr} upon the request of the \gls{tsu} for data and knowledge management so that they follow the data and knowledge management policy with regard to \gls{ilk} and \glspl{iplc} considerations.
    \end{itemize}

    \item \textbf{Assessment \gls{tsu}}
    \begin{itemize}
        \item Will collect the \glspl{dmr} from assessment \glspl{ipbesexpert};
        \item Will provide assistance to assessment \glspl{ipbesexpert} in making sure that the \glspl{dmr} adhere to this policy;
        \item Will provide a persistent identifier, such as a \gls{doi} for each \gls{datadepositpackage};
        \item Will add specific and consistent keywords and metadata to the \gls{datadepositpackage} (e.g., chapter, assessment, figure) to make \gls{data} findable and identifiable;
        \item Will make sure that their \glspl{ipbesexpert} fulfil their responsibilities as outlined in this policy and in the \glspl{dmr} and will collect the metadata of the \glspl{ipbesproduct} from the \glspl{ipbesexpert} so that it can be handed over to the \gls{tf} on data and knowledge management;
        \item Will develop and maintain, under the guidance of the \gls{tsu} on \gls{data} and \gls{knowledge} management, and in cooperation with the \gls{secretariat}, guidelines, tutorials, \glspl{workflow}, and examples (\glspl{dmr}) to enable \glspl{ipbesexpert} to implement this policy;
        \item Will collect and provide feedback to the {technical support unit on data and knowledge management} on lessons learned during the implementation of the \gls{data} and \gls{knowledge} management policy.
    \end{itemize}

    \item \textbf{\glspl{ipbesexpert}}
    \begin{itemize}
        \item Will prepare and keep up to date \glspl{dmr} for their IPBES-related \gls{knowledgesynthesis}. These \glspl{dmr} should be available at the latest at the first \gls{milestone}, and be updated for each following \gls{milestone}. The \glspl{dmr} should conform with this policy and follow the examples in the technical guidelines;
        \item Are responsible for fulfilling on the implementation of the \glspl{dmr} to their assessment  \gls{tsu};
        \item With a particular focus on ethics, transparency and validation, the \glspl{ipbesexpert} are responsible for the responsible use of \gls{ai}.
    \end{itemize}
\end{enumerate}

\section{AI Code of Practice}
\label{sec:aicop}

Placeholder for the AI CoP - at the moment in separate document.

\section{Implementation Resources}

\subsection{IPBES Data and Knowledge Management Tutorials}

The task force on data and knowledge management has created a series of tutorials focused on data and knowledge management to facilitate the implementation of the IPBES data and knowledge management policy. These short 3-10 minute tutorials cover the following topics: the data and knowledge management policy (formally known as the data management policy), data management reports, active research data, tools, and examples.

The tutorials can be found on the IPBES website:

\subsection{IPBES Technical Guidelines}

The technical support unit on data and knowledge management has produced a series of technical guidelines on data and knowledge management, handling, and delivery to provide detailed information and recommendations on specific topics such as cartographic elements for maps or file formats. These guidelines have been reviewed by the task force on data and knowledge management and serve as an important resource for assessment experts and technical support units.

The technical guidelines can be found on the IPBES ICT portal:


\end{document}